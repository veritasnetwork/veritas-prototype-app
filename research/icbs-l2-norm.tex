\documentclass[12pt]{article}
\usepackage{amsmath}
\usepackage{amssymb}
\usepackage{geometry}
\usepackage{hyperref}

\geometry{margin=1in}

\setcounter{secnumdepth}{0}

\title{The Geometry of Belief: Understanding the Euclidean Norm ICBS}
\author{}
\date{}

\begin{document}

\maketitle

\section{Markets as Instruments of Understanding}

How do you price the relevance of an emerging idea? Or the significance of a contested claim? Standard prediction markets require objective outcomes. But most information worth tracking (early signals, cultural shifts, subjective importance) has no clear ground truth.

The Inversely Coupled Bonding Surface (ICBS) \cite{icbs2020} provides the foundation for such markets. Traders compete by taking LONG or SHORT positions on content relevance. However, without external grounding, even coupled markets risk becoming purely reflexive, rewarding hype over insight. We have adapted the original ICBS design to support periodic settlement via the rebasing of virtual reserves. At regular intervals, reserves rebase according to crowd wisdom elicitation. In Veritas, we use Belief Decomposition scored via Bayesian Truth Serum. By anchoring to collective epistemic signals, prices become less reflexive and instead express continuous, honest crowd wisdom. This rewards being ``early and right'' while grounding speculation in shared understanding of relevance.

\section{Inverse Coupling}

In traditional bonding curves, buying token A doesn't affect token B's price. ICBS couples them: buying LONG pushes SHORT's price down, and vice versa. The Euclidean norm cost function $C(s_L, s_S) = \lambda\sqrt{s_L^2 + s_S^2}$ implements this through geometric distance, producing circular iso-cost curves where movement along any direction changes both prices.

\section{The Cost Function}

\subsection{Core Formula}

The Euclidean norm ICBS uses a simple cost function:

\begin{equation}
C(s_L, s_S) = \lambda \cdot \sqrt{s_L^2 + s_S^2}
\end{equation}

where $s_L$ and $s_S$ are the supplies of LONG and SHORT tokens, and $\lambda$ is a fixed scaling constant. Geometrically, this measures the Euclidean distance from the origin in the $(s_L, s_S)$ plane. Every point on a circle of radius $r$ costs $\lambda \cdot r$.

\subsection{Lambda: The Unit of Account}

At deployment, $\lambda$ is derived from the initial deposit $D$:

\begin{equation}
\lambda = \frac{D}{\sqrt{s_L^2 + s_S^2}}
\end{equation}

This fixes $\lambda$ permanently, establishing a consistent unit of account. Markets of different sizes have identical percentage-based price dynamics, enabling cross-market comparison.

\subsection{Marginal Prices}

Prices emerge as partial derivatives, the slopes of the cost landscape:

\begin{align}
p_L &= \frac{\partial C}{\partial s_L} = \lambda \cdot \frac{s_L}{\sqrt{s_L^2 + s_S^2}} \\
p_S &= \frac{\partial C}{\partial s_S} = \lambda \cdot \frac{s_S}{\sqrt{s_L^2 + s_S^2}}
\end{align}

Each token's price increases with its own supply but is suppressed by the opposing side, the essence of inverse coupling.

\subsection{Reserves and the Invariant}

Virtual reserves are defined as $r = s \cdot p$:

\begin{align}
r_L &= s_L \cdot p_L = \lambda \cdot \frac{s_L^2}{\sqrt{s_L^2 + s_S^2}} \\
r_S &= s_S \cdot p_S = \lambda \cdot \frac{s_S^2}{\sqrt{s_L^2 + s_S^2}}
\end{align}

The total value locked becomes:

\begin{equation}
\text{TVL} = r_L + r_S = \lambda \cdot \sqrt{s_L^2 + s_S^2}
\end{equation}

This equals the cost function exactly. This \textbf{on-manifold property} ensures solvency: total claimable value always matches what the vault holds. During settlement, reserves can rebase while preserving the invariant, enabling reward redistribution without minting or burning tokens.

\section{Properties and Implications}

The Euclidean norm ICBS has several key properties that make it well-suited for speculative belief markets:

\noindent\textbf{1. Inverse Coupling.} The two sides compete directly. Mathematically:
\begin{equation}
\frac{\partial p_L}{\partial s_S} = -\lambda \cdot \frac{s_L s_S}{(s_L^2 + s_S^2)^{3/2}} < 0
\end{equation}
Buying SHORT directly lowers LONG's price, and vice versa. This creates genuine opposition between beliefs rather than independent liquidity pools.

\noindent\textbf{2. Self-Scaling Liquidity.} Every trade moves supply further from the origin in $(s_L, s_S)$ space. Since the cost function measures Euclidean distance from the origin, and TVL equals the cost function exactly, TVL grows automatically as traders express conviction. No external liquidity providers needed. Markets bootstrap from minimal deposits and scale organically with trading volume. This differs fundamentally from LMSR, where the subsidy parameter $b$ bounds maximum loss and caps liquidity, and from AMMs, where the constant product invariant requires LPs to deposit capital to increase liquidity. In ICBS, conviction itself creates liquidity.

\noindent\textbf{3. Geometric Simplicity.} Iso-cost curves are circles: every point at distance $r$ from the origin costs $\lambda \cdot r$. This makes the mechanism intuitive and the math tractable. Only square roots, no fractional powers or exponentials.

\noindent\textbf{4. Fixed Unit of Account.} Since $\lambda$ is fixed at deployment, prices scale consistently. A \$100 pool and a \$10,000 pool have the same percentage price impact for equivalent percentage trades. This enables:
\begin{itemize}
\item Cross-market comparison of belief strength
\item Predictable trading dynamics as markets grow
\item Meaningful aggregation across multiple beliefs
\end{itemize}

\noindent\textbf{5. Arbitrarily Large Returns.} Prices range from 0 to $\lambda$:
\begin{equation}
\lim_{s_L \to \infty, s_S \text{ fixed}} p_L = \lambda
\end{equation}
Early backers who buy near zero can see prices approach $\lambda$, yielding arbitrarily large (though bounded by $\lambda$) returns as conviction accumulates. Unlike LMSR's fixed [0,1] probability bounds, this rewards discovery of undervalued ideas rather than just tracking consensus.

\noindent\textbf{6. Settlement Compatibility.} The on-manifold property (TVL always equals cost function) ensures solvency through settlement. Combined with virtualization, reserves can rebase according to external signals without minting or burning tokens. This grounds speculation in collective truth.

\section{Settlement: Anchoring Speculation in Truth}

Without external grounding, ICBS markets risk devolving into reflexive, hyper-speculative Keynesian beauty contests that reward only timing and hype, not insight. Settlement via Belief Decomposition (BD) scores anchors prices to collective epistemic signals.

\subsection{Settlement Mechanism}

At epoch end, an external signal $x \in [0,1]$ (the BD relevance score) determines how reserves rebase. The market's current prediction is:
\begin{equation}
q = \frac{r_L}{r_L + r_S}
\end{equation}

Settlement factors scale each side's reserves based on prediction error:
\begin{align}
    f_L &= \frac{x}{q} \\
    f_S &= \frac{1-x}{1-q}
\end{align}

If $x > q$, LONG was correct ($f_L > 1$, $f_S < 1$), and vice versa.

\subsection{Virtualization via Square-Root Scaling}

Settlement operates through scale parameters $\sigma$ that convert display supplies to virtual supplies: $s_{\text{virtual}} = s_{\text{display}} / \sigma$. These scale by the square root of settlement factors:
\begin{align}
    \sigma'_L &= \frac{\sigma_L}{\sqrt{f_L}} \\
    \sigma'_S &= \frac{\sigma_S}{\sqrt{f_S}}
\end{align}

Why square roots? Because prices depend on ratios of virtual supplies. This scaling converges the market prediction toward $x$ without minting or burning tokens.

Reserves scale directly by settlement factors:
\begin{align}
    r'_L &= r_L \cdot f_L \\
    r'_S &= r_S \cdot f_S
\end{align}

Capital flows from the incorrect side to the correct side. Total vault balance remains constant. Token holders gain or lose purchasing power based on prediction accuracy.

\begin{thebibliography}{9}

\bibitem{icbs2020}
Nick Williams and Vitalik Buterin,
\textit{Better Curation via Inversely Coupled Bonding Surfaces},
Ethereum Research, 2020.
\url{https://ethresear.ch/t/better-curation-via-inversely-coupled-bonding-surfaces/7613/2}

\end{thebibliography}

\end{document}
